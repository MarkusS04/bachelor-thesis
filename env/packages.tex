% Encoding, Sprache, Schrift
\usepackage[ngerman]{babel}
\usepackage[T1]{fontenc}
\usepackage[utf8]{inputenc}
\usepackage{csquotes}
\renewcommand{\rmdefault}{ptm}

% Bilder einbinden
\usepackage{graphicx}
\graphicspath{ {picture/} }

% Seitenränder und sonstige Formate festlegen
\usepackage[a4paper,left=3cm,right=2cm,top=3cm,bottom=2.5cm]{geometry}
\usepackage{setspace}

%------------- Referenzen und Verzeichnisse -------------%
% Referenzen und URLs
\usepackage{url}
\usepackage{hyperref}
\usepackage{hyperxmp}
\hypersetup{
  colorlinks   = true,
  urlcolor     = black,
  linkcolor    = black,
  citecolor    = black
}
% Abbildungs + Tabellenverzeichnis
\usepackage[nottoc,numbib]{tocbibind}
% Listingverzeichnis
\usepackage{listings}
\renewcommand{\lstlistlistingname}{Listingverzeichnis}
% "Abb.", "Tbl." und "Lst."
\ifdefined\varShowTitlesInLists
  \makeatletter
  \renewcommand{\l@figure}[2]{\@dottedtocline{1}{1.5em}{2.3em}{Abb. #1}{#2}}
  \renewcommand{\l@table}[2]{\@dottedtocline{1}{1.5em}{2.3em}{Tbl. #1}{#2}}
  \renewcommand{\l@lstlisting}[2]{\@dottedtocline{1}{1.5em}{2.3em}{Lst. #1}{#2}}
  \makeatother
\fi
\newcommand{\varShowTitlesInLists}{true}
% Akronyme
\usepackage[acronym, nogroupskip, nonumberlist, nopostdot]{glossaries}
\loadglsentries{env/acronym.tex}
\makenoidxglossaries
\setacronymstyle{long-sc-short}
% Fußnoten
\usepackage[hang]{footmisc}
\renewcommand{\footnotemargin}{12pt}
% Unterschriften Bild + Tabelle
\usepackage[font=small, justification=centering]{caption}
% Literaturverzeichnis
\usepackage[style=ext-authoryear,     % ext- ermöglicht das Einblenden der Klammern um die Jahreszahl in Fußnoten
            sorting=nyt,              % Nach Nachnamen des ersten genannten Autorens sortieren, dann Jahr, dann Titel
            isbn=false,               % Ausblenden des Feldes
            url=false,                % Ausblenden des Feldes
            doi=true,                 % Einblenden des Feldes
            eprint=false,             % Ausblenden des Feldes
            clearlang=true,           % Ausblenden von Sprachcodes
            maxcitenames=2,           % Ab drei Autoren mit "et at." abkürzen
            citexref=true,            % Aufsplitten von Referenz in Sammelwerken
            mincrossrefs=0,           % Aufsplitten direkt bei der ersten Referenz auf Sammelwerk
            backref=true,             % Rückreferenzen vom Literaturverzeichnis in den Text
            maxbibnames=100]{biblatex}% Alle Autoren im Literaturverzeichnis ausschreiben
\addbibresource{content/references.bib}
\setcounter{biburllcpenalty}{1000}
% Ausblenden der Klammern um die Jahresangabe im Literaturverzeichnis
\ifdefined\varNoParenthesesAroundYear
  \makeatletter
  \def\act@on@bibmacro#1#2{%
    \expandafter#1\csname abx@macro@\detokenize{#2}\endcsname
  }
  \def\patchbibmacro{\act@on@bibmacro\patchcmd}
  \def\pretobibmacro{\act@on@bibmacro\pretocmd}
  \def\apptobibmacro{\act@on@bibmacro\apptocmd}
  \def\showbibmacro{\act@on@bibmacro\show}
  \makeatother

  \patchbibmacro{date+extradate}{%
  \printtext[parens]%
  }{%
  \setunit{\addperiod\space}%
  \printtext%
  }{}{}
\fi
%------------- Referenzen und Verzeichnisse -------------%

%--------------- Seitenspezifische imports --------------- %
% für Titelseite
\usepackage{chngpage}
\usepackage{calc}
\usepackage{float}
% Code
\usepackage{listings}
\usepackage{multicol}
\usepackage{wrapfig}
\usepackage{framed}
\usepackage[table]{xcolor}
%--------------- Seitenspezifische imports --------------- %


%-------------- sonstiges - nicht verwendet --------------%
% \usepackage{booktabs}
% \usepackage{enumitem}
% \usepackage[figure,table,lstlisting]{totalcount}