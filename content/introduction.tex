\section{Einführung}
\subsection{Allgemein}
Die \mediserv{} bietet den verschiedenen Gruppen von Ärzten und Kliniken eine Lösung zur Privatabrechnung und Finanzdienstleistungen.
Dabei werden Rechnungen angekauft, verarbeitet und an den Rechnungsempfänger versandt.
Da Gesundheitsdaten, welche besondere personenbezogene Daten sind,
besonders schützenswerte Informationen sind, werden die Rechnungen per Post versendet. \vgl{eu_dsgvo_2016}{EU-DSGVO Art. 1, Art. 4 Satz 1 und 15, Art. 9}
Qualität ist heute ein wichtiges Merkmal in sämtlichen Bereichen eines Unternehmens, so auch im \acrfull{OPM}.
Deshalb soll es im Rahmen dieser Bachelorarbeit ein softwaregestützter Prozess erarbeitet werden,
um sowohl die Qualität der erstellten Dokumente vor dem Versand sicherzustellen,
aber auch vorgelagerte manuelle Prozesse besser digital zu unterstützen.


\subsection{Motivation}
Durch eine immer größer werdende Anzahl an angekauften Rechnungen erhöht sich gleichzeitig der Aufwand im Fachbereich.
Neben der Bearbeitung von Anfragen von Kunden und Rechnungsempfänger gilt es dabei auch die Interessen der \mediserv{} zu wahren.
Dazu zählt unter anderem die Unterstützung neuer Kunden
sowie die Kontrolle von abgerechneten Rechnungen, um Betrug von Ärzten zu verhindern und damit auch Reklamationen durch Rechnungsempfänger zu verringern.
Diese Prozesse sind aktuell durch Systembrüche ausgezeichnet und erfordern viele manuelle Schritte.
Um den Aufwand zu reduzieren, soll deshalb ein softwaregestützter Prozess erarbeitet werden, welche für Entlastung und weniger Fehlerpotential sorgt.

\subsection{Zielsetzung}
Das Ziel dieser Arbeit ist, einen digitalen Prozess zu konzipieren und dabei zu evaluieren,
ob eine Standardsoftware den Anforderungen gerecht wird, oder ob eine Eigenentwicklung besser geeignet ist.
Zusätzlich soll die Umsetzung des Projekts und die Einführung geplant werden.
\subsection{Aufbau der Arbeit}
Um das Ziel der Arbeit zu erreichen werden folgende Themen in der Arbeit beleuchtet:
\\ [10pt] Analyse der aktuellen Prozesse - Darstellung, wie die Prozesse innerhalb der Fachabteilung aktuell aufgebaut sind
\\ [10pt] Analyse der Anforderungen - Darstellung, was die Fachabteilung in einem neuen, automatischem und mit weniger Systembrüchen ausgestaltetem Prozess, benötigt
\\ [10pt] Technische Anforderungen - Anforderungen an den Datenschutz, Berechtigungen und Sicherheit der Software
\\ [10pt] Technische Grundlagen - Beschreibung der im Unternehmen verwendeten Technologien und deren Interaktion
\\ [10pt] Analyse der Standardsoftware - Beschreibung welche Funktionalität die Software bietet, welche Anforderungen sie erfüllt bzw. was fehlt, und ob die Prozesse entsprechend angepasst werden könnten
\\ [10pt] Konzeption einer Eigenentwicklung - Darstellung eines Konzepts für eine eigene Software, die alle technischen und fachlichen Anforderungen erfüllt und wie es in den Prozessablauf integriert werden kann
\\ [10pt] Vergleich Standardsoftware mit Eigenentwicklung - Beurteilung der Vor- und Nachteile beider Wege, Entscheidungsvorschlag welches Tool zu wählen ist
\\ [10pt] Fazit und Ausblick - Zusammenfassung der Ergebnisse der Arbeit und Beschreibung der Umsetzung des Projekts